\chapter{Introducción}

\section{Motivación}

En el área de robótica, el problema de la exploración consiste en recorrer, mediante un robot autónomo, una zona 
desconocida para obtener el conocimiento total de esta, o en caso de no poder reconocerla en su totalidad, maximizar 
el espacio explorado. 


La utilización de robots es muy importante para la cartografía o búsqueda y rescate en lugares que son peligrosos o 
inaccesibles para las personas. Hay muchos ejemplos de exploración con robots autónomos, como las sondas espaciales 
no tripuladas que exploran lugares antes de que lleguen los astronautas, o robots que ingresan en estructuras colapsadas 
para crear una reconstrucción del entorno y reconocer exactamente el lugar en el que se encuentran las víctimas.


Los robots autónomos necesitan un mapa para poder operar en un entorno particular. Por esta razón, se utiliza la técnica 
de localización y modelado simultáneos, para construir un mapa de una zona desconocida en la que se encuentra el robot, 
a la vez que estima su trayectoria al desplazarse dentro de la misma.


Existen trabajos previos que logran generar un mapa mediante exploración, trabajando sobre entornos estáticos. 
Se utilizan diversas técnicas para el aprendizaje sobre el entorno, como por ejemplo redes neuronales \cite{TP2}, 
grafos \cite{TP4} o múltiples robots \cite{TP5}. Algunos trabajos dejan abierta la pregunta sobre como explorar 
entornos dinámicos \cite{TP1} \cite{TP3}. También existen trabajos previos que contemplan cambios en el entorno \cite{TP6}.


En este trabajo estamos interesados en decidir si es posible alcanzar un 
objetivo (posición) en el mapa evitando lo maximo posible que el robot explore 
la totalidad del mapa. 
Dicho problema fue explorado 
en~\cite{melchior2007particle} ampliando el algoritmo tradicional que otorga 
un camino hacia el objetivo para que soporte incertidumbre del area a recorrer. 

\section{Resumen de la contribución}

Las contribuciones de esta tesis pueden verse desde un punto de vista teórico. 
Durante los últimos años, para resolver problemas de alcanzabilidad de un 
objetivo en una area, se han usado algoritmos basados en camino mínimo como 
Dijkstra, A*, entre otros. Sin embargo, este trabajo queda como evidencia de 
que nuevas estrategias pueden usarse. Decidimos utilizar algoritmos de síntesis 
de controladores para poder resolver el problema descrito.

Por otro lado, las técnicas existentes de síntesis de controladores nos 
permiten hallar una estrategia para el robot cuando el area a transitar es 
totalmente conocida. Para casos donde esta suposición del ambiente es muy 
fuerte, actualmente, no existe forma de obtener un plan adecuado. La forma 
natural de modelar un area parcialmente conocida es mediante el modelado de un 
MTS (Modal Transition System), formalismo matemático que introduciremos en la 
sección~\ref{sec:MTS}. 

Previo a nuestro trabajo, la síntesis de controladores para dichos ambientes 
genera una respuesta trivaluada: ``Sin importar cómo se resuelvan las 
incertidumbre del ambiente, SIEMPRE obtendremos un plan para llegar al 
objetivo'', ``Sin importar cómo se resuelvan las incertidumbres del ambiente, 
NUNCA obtendremos un plan para llegar al objetivo'' o ``depende de cómo se 
resuelvan las incertidumbres''. Como contribución, extenderemos esta respuesta 
para que, en los casos favorables, además de asegurarnos la existencia, nos de 
el plan para llegar al objetivo.

En resumen, lo que queremos es poder dar una respuesta certera a si es posible 
garantizar 
el cumplimiento del objetivo. 
Para lograrlo necesitamos explorar para adquirir nueva información en caso de 
que la información que poseemos no sea 
suficiente para dar una respuesta. Extenderemos la herramienta MTSA (Modal 
Transition System Analyser) para dar soporte 
a la exploración y presentaremos una estrategia que nos permita dar una 
respuesta certera sobre la posibilidad de 
cumplimiento del objetivo.


\section{Esquema de la tesis}

A continuación, en el capítulo 2, introduciremos la teoría sobre la cual se 
sustenta nuestra 
solución al problema. 
En este capítulo definiremos formalmente la matemática necesarias que 
posteriormente vamos a utilizar. Luego, en el capítulo 3, introduciremos, 
formalmente, el problema a resolver. Presentaremos un algoritmo que lo resuelve 
y una posible estrategia de exploración que garantice encontrar una respuesta a 
este problema. El capítulo 4 muestra resultados de nuestra técnica.
Para ello, mostraremos como se comporta nuestro algoritmo y nuestra estrategia 
frente a una serie de problemas que servirán para mostrar las características 
de nuestra implementación. Despues en el capítulo 5, daremos conclusiones de 
este trabajo. Analizaremos el algoritmo y la estrategia, y plantearemos que 
trabajos a futuro podrían hacerse a partir de esta tesis. 
Finalmente, en el capítulo 6 y a modo de apendice, adjuntamos el manual de 
usuario para exploración en la herramienta MTSA basado en un ejemplo.
