%\begin{center}
%\large \bf \runtitulo
%\end{center}
%\vspace{1cm}
\chapter*{\runtitulo}

\noindent
En el área de robótica, el problema de la exploración consiste en recorrer, mediante un vehículo autónomo, 
una zona desconocida para obtener conocimiento sobre ella. La utilización de robots es muy importante para 
la cartografía o búsqueda y rescate en lugares que son peligrosos o inaccesibles para las personas. 
En el caso de la utilización de robots autónomos, se utiliza la técnica de localización y modelado simultáneo, 
para construir un mapa de la zona desconocida en la que se encuentra el robot, a la vez que estima su 
trayectoria al desplazarse dentro de la misma.

\vspace{\baselineskip}
El objetivo de esta tesis es resolver el problema de localización y modelado simultáneo utilizando como modelo 
matemático los MTSs (Modal Transition Systems), los cuales son nociones abstractas de los LTSs (Labelled Transition Systems). 
Dado un conjunto de objetivos y un conjunto de suposiciones de dominio, los MTSs nos permiten determinar, mediante la 
síntesis de controladores, si los objetivos son imposibles de alcanzar, si podemos garantizar su cumplimiento, o si 
las suposiciones de dominio son insuficientes para decidir sobre los objetivos.

\vspace{\baselineskip}
Para cumplir con el objetivo de la tesis, extenderemos la herramienta MTSA (Modal Transition System Analyser) para 
dar soporte a la exploración, y presentaremos una estrategia, que bajo ciertas condiciones, nos permita ir agregando 
información a nuestras suposiciones de dominio, hasta poder decidir si es o no posible garantizar el cumplimiento de 
los objetivos de la exploración.

\bigskip

\noindent\textbf{Palabras claves:} Exploración, Modelado, LTS, MTS, Síntesis de controladores.