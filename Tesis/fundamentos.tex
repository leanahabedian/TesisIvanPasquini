\chapter{Fundamentos teóricos}

\section{El mundo y la maquina}
El modelo del mundo y la máquina de Michael Jackson es una abstracción que permite formular rigurosamente algunas nociones
fundamentales de la ingeniería de requerimientos. En este modelo, los fenómenos son hechos, situaciones o eventos cuya
existencia puede observarse, la maquina es una porción del sistema a desarrollar o modificar y el mundo es una porción de
la realidad afectada por la máquina. El mundo y la máquina interactúan en la interfaz de la máquina. Un problema del mundo 
describe una parte del mundo real que nosotros queremos mejorar construyendo la solución como una máquina.

\vspace{\baselineskip} 
En este modelo, los requerimientos R son declaraciones prescriptivas (propiedades deseadas que pueden cumplirse o no) sobre 
el mundo expresadas en términos de fenómenos sobre la interfaz entre la máquina que queremos construir y el mundo en el cual 
viven los problemas que queremos resolver. Los requerimientos deben ser forzados por la maquina independientemente de cómo 
se comporte el mundo. Los problemas del mundo son capturados como declaraciones prescriptivas expresadas en término de 
fenómenos del mundo llamados objetivos G, y declaraciones descriptivas (propiedades sobre el sistema que se mantienen 
independientes a cómo se comporta el sistema) sobre lo que nosotros asumimos que es verdad en el mundo, suposiciones 
de dominio D. Las suposiciones de dominio podrían no mantenerse y deben ser satisfechas por el mundo. 

\vspace{\baselineskip}
La tarea clave de la ingeniera de requerimientos es comprender y documentar los objetivos y las características del entorno.
Teniendo estos modelos se pueden formular un conjunto de requerimientos para la máquina de forma que en conjunto con el
entorno satisfagan los objetivos. Más formalmente R, D $\vDash$ G.
Esto puede ser formulado como un problema de síntesis. Dados un conjunto de suposiciones del dominio y un conjunto de
objetivos del sistema, construir  automáticamente un modelo operacional de la maquina tal que compuesto con el modelo del
entorno, los objetivos sean alcanzados.

\vspace{\baselineskip}
Diremos que un evento es controlable si es controlable por la máquina. Una evento es monitoreable o no controlable si es
controlable por el mundo. 
Para cumplir los objetivos, el modelo operacional restringe la ocurrencia de eventos controlables basándose en la observación
de los eventos que ya ocurrieron.
Este problema se conoce como el problema de síntesis de controladores y está siendo estudiado exhaustivamente en varios
aspectos de la ingeniería de los requerimientos.

\section{Labelled Transition Systems}
Los Labelled Transition Systems (LTSs) son ampliamente utilizados para modelar y analizar el comportamiento de sistemas
concurrentes y distribuidos. Son un sistema de transición de estados en el cual las transiciones son etiquetadas con
acciones. El conjunto de acciones de un LTS se conoce como alfabeto comunicacional y constituye las interacciones que
el sistema modelado puede tener con el entorno.

\vspace{\baselineskip}
Sea $Estados$ el conjunto universal de estados y $Acciones$ el conjunto universal de etiquetas de acciones.
Un Labelled Transition System es una tupla $E = (S_{E}, A_{E}, \Delta_{E}, s_{0})$ en donde
$S_{E} \subseteq Estados$ es un conjunto finito de estados, $A_{E} \subseteq Acciones$ es un alfabeto finito,
$\Delta_{E} \subseteq (S_{E}$ x $A_{E}$ x $S_{E})$ es una relación de transición y $s_{0} \subseteq S_{E}$ 
es el estado inicial.

\vspace{\baselineskip}
Dado $(s, l, s$’$) \subseteq \Delta_{E}$ decimos que $l$ está habilitada desde $s$ en $E$. Decimos que un LTS es
determinístico si desde un estado, al realizar una acción, hay solo un estado posible al que llegamos.

\section{Composición en paralelo}

Sean $M$ y $E$ dos LTSs. La composición en paralelo $\parallel$ es un operador simétrico que crea un nuevo LTS $M\parallel$$E$, 
tal que sus estados son el producto cartesiano de los estados de $M$ y $E$, sus acciones la unión de las acciones de $M$ y $E$, 
y su función de transición sincroniza el movimiento de las dos máquinas en las acciones compartidas, mientras que se 
mueven independientemente en sus acciones propias.

\section{LTS Legal}
Dados dos LTSs $E$ y $M$, y $A_{Eu} \subseteq A_{E}$, decimos que $M$ es un Legal LTS para $E$ respecto a $A_{Eu}$ si por cada
estado en $E\parallel$$M$, realizar una acción perteneciente a $A_{Eu}$ es igual a realizar la acción en $E$.
Intuitivamente, en la composición una acción es deshabilitada si y solo si también es deshabilitada en el LTS original.
En otras palabras, $M$ no restringe a $E$ respecto a $A_{Eu}$.

\section{Traza}

Dado un LTS $E$, una secuencia $\pi = l_{0}, l_{1}, ...$ es una traza en $E$ si existe una secuencia 
$s_{0}, l_{0}, s_{1}, l_{1}, s_{2}, ...$ en donde $(s_{i}, l_{i}, s_{i + 1})$ está en la función de transición de $E$.

\section{Modal Transition System}
Los MTS (Modal Transition System), son nociones abstractas de los LTSs. Extienden a los LTSs ya que las transiciones 
que los MTSs poseen, pueden denotar eventos requeridos o bien posibles. Debido a esta particularidad de los MTSs, es 
que podemos modelar mediante este formalismo información parcial del mundo.

\vspace{\baselineskip}
La definición es la misma que para LTSs, pero la función de transición se divide en dos para separar las acciones
requeridas de las posibles. Las posibles incluyen a las requeridas.
Más formalmente, un MTS es una quintupla compuesta por $(S, A, \Delta^r, \Delta^p, s_{0})$ en donde $S$ es el
conjunto de estados, $A$ el conjunto de acciones, $\Delta^r$ las transiciones requeridas, $\Delta^p$  las
transiciones posibles y $s_{0}$ el estado inicial.

\section{Refinamiento}

Hay una relación de refinamiento entre los MTSs y los LTSs. Un LTS se puede ver como un MTS en donde la función de
transición de las acciones posibles es igual que la función de transición de las acciones requeridas. Los LTSs que
refinan un MTS son descripciones completas del comportamiento del sistema y se llaman implementaciones.

\vspace{\baselineskip}
Sean $E$ y $N$ dos MTSs. $N$ refina a $E$ si cada transición requerida de $E$ son requeridas en $N$ y cada posible
transición de $N$ también es posible en $E$.

\section{Implementación}

Un LTS $L$ es una implementación de un MTS $M$ si y solo si $L$ refina a $M$. Una implementación es deadlock free
si todos sus estados tienen transiciones salientes. Un MTS es determinístico si ninguna de sus acciones conduce
desde un estado a más de un estado.

\section{Fluent Linear Temporal Logic}
Los fluents nos permiten especificar propiedades basadas en estados, en modelos basados en eventos. FLTL es una
lógica temporal lineal que nos permite razonar sobre fluents. Un fluent es de la forma $Fl = <I_{fl}, T_{fl}, Init_{fl}>$.
$I_{fl}$ y $T_{fl}$ pertenecen al conjunto de acciones. $I_{fl}$ son las acciones que inicializan, $T_{fl}$ las que
terminan y $Init_{fl}$ indica el estado inicial.

\vspace{\baselineskip}
Una formula FLTL está compuesta por la lógica de primer orden y los operadores temporales $X$ (next) y $U$ (strong until).
También se utilizan $\Diamond$ (eventually) y $\square$ (always) como azúcar sintáctico.
Dada una traza, por cada posición en la traza podemos decidir si el fluent está activo o no.

\vspace{\baselineskip}


\section{Generalised Reactivity (1)}

Dada una secuencia infinita de estados, una formula GR(1) denota cuales han ocurrido infinitamente. 
Más formalmente, $\phi = (A, B)$ en donde $A$ y $B$ son conjuntos de subconjuntos de estados en donde o bien alguno de los
subconjuntos de $A$ no contiene ningún estado que se visita infinitamente, o bien todos los subconjuntos de $B$ contienen 
al menos un estado que se visita infinitamente.

\section{Síntesis de controladores}

La síntesis de controladores se encarga de producir automáticamente una máquina que restringe la ocurrencia de eventos
controlables basándose en la observación de los eventos que ya ocurrieron. Cuando esta máquina es desplegada en un ambiente
adecuado garantiza la satisfacción de un conjunto de objetivos dado. La satisfacción de estos objetivos depende de la
satisfacción de las asunciones sobre el entorno.

\vspace{\baselineskip}
Podemos describir la síntesis de controladores de la siguiente forma. Dados un LTS que describe el comportamiento
del entorno, un conjunto de acciones controlables, un conjunto de fórmulas FLTL que representan las suposiciones
del dominio y un conjunto de fórmulas FLTL que representan los objetivos del sistema, el problema de control de
LTS es encontrar un LTS que solo restrinja la ocurrencia de acciones controlables y garantice que la composición
en paralelo entre el ambiente y el LTS es deadlock free y que si las suposiciones de dominio se satisfacen, entonces
los objetivos del sistema también son satisfechos.

\subsection{LTS control problem}

Dado un LTS determinístico $E$, un conjunto de acciones controlables $A_{c}$ y una formula FLTL $\phi$, una solución 
para el LTS control problem $\epsilon = <E, \phi, A_{c}>$ es un LTS $M$ legal para $E$ tal que para cada estado las
acciones son las mismas que para su estado correspondiente en $E$, $E\parallel$$M$ es deadlock free y toda traza en
$E\parallel$$M$ cumple $\phi$. El LTS Control Problem es decidible en tiempo exponencial y en caso de ser decidible,
se encuentra la solución con la misma complejidad. Para esto es necesario que el modelo sea determinístico.

\subsection{MTS	control problem}

El problema de la síntesis de controladores para MTSs consiste en ver si todas, alguna o ninguna de sus implementaciones
pueden ser controladas por un controlador LTS. Al responder esta pregunta, también estamos respondiendo la pregunta de
si el problema es realizable, ya que si la respuesta es ninguno el problema no es realizable. Para decidir este problema,
primero se intenta encontrar un controlador para la implementación del MTS que tiene todas las acciones posibles controlables.
Si se lo encuentra, se tiene un controlador para todas las posibles implementaciones. Se hace lo mismo con el que tiene la
mínima cantidad de acciones controlables y así se obtiene la respuesta.