\chapter{Fundamentos teóricos}

\section{El mundo y la máquina}
El modelo del mundo y la máquina de Michael Jackson \cite{MundoMaquina} es una abstracción que permite formular rigurosamente 
algunas nociones fundamentales de la ingeniería de requerimientos. En este modelo, los fenómenos son hechos, situaciones o 
eventos cuya existencia puede observarse, la máquina es una porción del sistema a desarrollar o modificar y el mundo es una 
porción de la realidad afectada por la máquina. El mundo y la máquina interactúan en la interfaz de la máquina. Un problema 
del mundo describe una parte del mundo real que nosotros queremos mejorar construyendo la solución como una máquina.


En este modelo, los requerimientos R son declaraciones prescriptivas (propiedades deseadas que pueden cumplirse o no) sobre 
el mundo expresadas en términos de fenómenos sobre la interfaz entre la máquina que queremos construir y el mundo en el cual 
viven los problemas que queremos resolver. Los requerimientos deben ser forzados por la máquina independientemente de cómo 
se comporte el mundo. Los problemas del mundo son capturados como declaraciones prescriptivas expresadas en término de 
fenómenos del mundo llamados objetivos G, y declaraciones descriptivas (propiedades sobre el sistema que se mantienen 
independientes a cómo se comporta el sistema) sobre lo que nosotros asumimos que es verdad en el mundo, suposiciones 
de dominio D. Las suposiciones de dominio podrían no mantenerse y deben ser satisfechas por el mundo.


La tarea clave de la ingeniería de requerimientos es comprender y documentar los objetivos y las características del entorno. 
Teniendo estos modelos se pueden formular un conjunto de requerimientos para la máquina de forma que en conjunto con el 
entorno satisfagan los objetivos. Más formalmente R, D $\vDash$ G. 
Esto puede ser formulado como un problema de síntesis \cite{Sintesis}. Dados un conjunto de suposiciones del dominio y un conjunto de 
objetivos del sistema, construir  automáticamente un modelo operacional de la máquina tal que compuesto con el modelo del 
entorno, los objetivos sean alcanzados.


Diremos que un evento es controlable si es controlable por la máquina. Un evento es observable, o no controlable, si es 
controlable por el mundo. 
Para cumplir los objetivos, el modelo operacional restringe la ocurrencia de eventos controlables basándose en la observación 
de los eventos que ya ocurrieron. 
Este problema se conoce como el problema de síntesis de controladores y está siendo estudiado exhaustivamente en varios 
aspectos de la ingeniería de los requerimientos.

\section{Labelled Transition Systems}
\label{sec:LTS}

Los Labelled Transition Systems \cite{LTS} (LTSs) son ampliamente utilizados para modelar y analizar el comportamiento de 
sistemas concurrentes y distribuidos. Son un sistema de transición de estados en el cual las transiciones son etiquetadas 
con acciones. El conjunto de acciones de un LTS se conoce como alfabeto comunicacional y constituye las interacciones que 
el sistema modelado puede tener con el entorno.

\begin{definition}{(Labelled Transition System)}
Sea $Estados$ el conjunto universal de estados y $Acciones$ el conjunto universal de etiquetas de acciones. Un Labelled 
Transition System es una tupla $E = (S_{E}, A_{E}, \Delta_{E}, s_{0}^{E})$ en donde $S_{E} \subseteq Estados$ es un conjunto 
finito de estados, $A_{E} \subseteq Acciones$ es un alfabeto finito, $\Delta_{E} \subseteq (S_{E} \times A_{E} \times S_{E})$ es una 
relación de transición y $s_{0}^{E} \in S_{E}$ es el estado inicial.
\end{definition}

Dado $(s, l, s$’$) \subseteq \Delta_{E}$ decimos que $l$ está habilitada desde $s$ en $E$. Decimos que un LTS es 
determinístico si desde un estado, al realizar una acción, existe solamente un estado posible al que llegamos.

Sean $M$ y $E$ dos LTSs. La composición en paralelo $\parallel$ es un operador 
simétrico que crea un nuevo LTS $M \parallel E$, 
tal que sus estados son el producto cartesiano de los estados de $M$ y $E$, sus acciones la unión de las acciones de $M$ y $E$, 
y su función de transición sincroniza el movimiento de las dos máquinas en las acciones compartidas, mientras que se 
mueven independientemente en sus acciones propias.

\begin{definition}{(Composición en paralelo)}
Sean $M = (S_{M}, A_{M}, \Delta_{M}, s_{0}^{M})$ y\\
$N = (S_{N}, A_{N}, \Delta_{N}, s_{0}^{N})$ dos LTSs. Composición en paralelo $\parallel$ es un operador simétrico tal que 
$M \parallel N$ es el LTS $P = (S_{M} \times S_{N}, A_{M} \cup A_{N}, \Delta_{P}, (s_{0}^{M}, s_{0}^{N}))$, donde $\Delta_{P}$ es 
la menor relación que satisface las siguientes reglas:
\begin{center}
\vspace{\baselineskip}
$\dfrac{(s, l, s') \in \Delta_{M}}{((s, t), l, (s', t)) \in \Delta_{P}}$ con $l \in A_{M} \setminus A_{N}$

\vspace{\baselineskip}
$\dfrac{(t, l, t') \in \Delta_{N}}{((s, t), l, (s, t')) \in \Delta_{P}}$ con $l \in A_{N} \setminus A_{M}$

\vspace{\baselineskip}
$\dfrac{(s, l, s') \in \Delta_{M}, t, l, t') \in \Delta_{N}}{((s, t), l, (s', 
t')) \in \Delta_{P}}$ con $l \in A_{M} \cap A_{N}$
\end{center}
\end{definition}


Una traza es una secuencia de acciones ejecutadas en un LTS. Una traza puede 
ser finita o infinita.

\begin{definition}{(Traza)}
Sea $E = (S_{E}, A_{E}, \Delta_{E}, s_{0}^{E})$ un LTS. Una secuencia $\pi = 
l_{0}, l_{1}, ...$ es una traza en $E$ si existe una 
secuencia $s_{0}, l_{0}, s_{1}, l_{1}, ...$ en donde para todo $i \geq 0$ vale 
$(s_{i}, l_{i}, s_{i + 1}) \in \Delta_{E}$.
\end{definition}

La bisimulación es una relación de equivalencia entre LTSs.

\begin{definition}{(Bisimulación)}
Sea $L$ el universo de todos los LTSs. Una relación binaria $R \subseteq L 
\times L$ es una bisimulación si y sólo si cuando  
$(P, Q) \in R$ entonces para cada acción $a \in A_{P} \cup A_{Q}$:

\begin{itemize}

\item
$(P \xrightarrow{a} P') \Rightarrow (\exists Q' \cdot Q \xrightarrow{a} Q' 
\land (P', Q') \in R)$

\item
$(Q \xrightarrow{a} Q') \Rightarrow (\exists P' \cdot P \xrightarrow{a} P' 
\land (P', Q') \in R)$

\end{itemize}

\end{definition}


La equivalencia entre LTSs se define mediante la noción de bisimilaridad.

\begin{definition}{(Bisimilaridad)}
Sea $L$ el universo de todos los LTSs. Dos LTSs $P, Q \in L$ son bisimilares si 
y sólo si existe una bisimulación $R$ tal que $(P, Q) \in R$.

\end{definition}


\section{Modal Transition System}
\label{sec:MTS}

Los Modal Transition System~\cite{MTS} (MTS), son nociones abstractas de los 
LTSs. Extienden a los LTSs ya que las transiciones 
que los MTSs poseen, pueden denotar eventos requeridos o bien posibles. Debido a esta particularidad de los MTSs, es 
que podemos modelar mediante este formalismo información parcial del mundo.

\vspace{\baselineskip}
La definición es la misma que para LTSs, pero la función de transición se divide en dos para separar las acciones 
requeridas de las posibles. Las posibles incluyen a las requeridas.

\begin{definition}{(Modal Transition System)}
Sea $Estados$ el conjunto universal de estados y $Acciones$ el conjunto universal de etiquetas de acciones. Un Modal 
Transition System es una tupla $M = (S_{M}, A_{M}, \Delta_{M}^{r}, \Delta_{M}^{p}, s_{0}^{M})$ en donde $S_{M} \subseteq Estados$ 
es un conjunto finito de estados, $A_{M} \subseteq Acciones$ es un alfabeto finito, 
$\Delta_{M}^{r} \subseteq \Delta_{M}^{p} \subseteq (S_{M} \times A_{M} \times S_{M})$ son las relaciones de transición requeridas 
y posibles respectivamente, y $s_{0}^{M} \in S_{M}$ es el estado inicial.
\end{definition}

Hay una relación de refinamiento entre los MTSs y los LTSs. Un LTS se puede ver 
como un MTS en donde la función de 
transición de las acciones posibles es igual que la función de transición de las acciones requeridas. Los LTSs que 
refinan un MTS son descripciones completas del comportamiento del sistema y se llaman implementaciones.

\begin{definition}{(Refinamiento de MTS)}
Sean $M = (S_{M}, A, \Delta_{M}^{r}, \Delta_{M}^{p}, s_{0}^{M})$ y\\
$N = (S_{N}, A, \Delta_{N}^{r}, \Delta_{N}^{p}, s_{0}^{N})$ dos MTSs. La relación $H \subseteq S_{M} \times S_{N}$ es un refinamiento 
entre $M$ y $N$ si para todo $l \in A$, $(s_{M}, s_{N}) \in H$ se cumple:

\begin{itemize}
\item
Si $(s_{M}, l, s_{M}') \in \Delta_{M}^{r}$ entonces existe $s_{N}'$ tal que $(s_{N}, l, s_{N}') \in \Delta_{N}^{r}$ y $(s_{M}', s_{N}') \in H$.
\item
Si $(s_{N}, l, s_{N}') \in \Delta_{N}^{p}$ entonces existe $s_{M}'$ tal que $(s_{M}, l, s_{M}') \in \Delta_{M}^{p}$ y $(s_{M}', s_{N}') \in H$.
\end{itemize}

Decimos que $N$ refina a $M$ si existe una relación de refinamiento $H$ entre $M$ y $N$ tal que $(s_{0}^{M}, s_{0}^{N}) \in H$, denotada como $M \preceq N$.

\end{definition}

Para facilitar el lenguaje, permitimos hablar de la relación inversa entre MTSs 
llamada implementación.

\begin{definition}{(Implementación de MTS)}
Un LTS $L$ es una implementación de un MTS $M$ si y sólo si $M \preceq L$.
\end{definition}

Una implementación es deadlock free si todos sus estados tienen transiciones salientes. 
Un MTS es determinístico si ninguna de sus acciones conduce desde un estado a más de un estado.

\section{Fluent Linear Temporal Logic}
\label{sec:FLTL}

Linear Temporal Logic (LTL) es una logica ampliamente utilizada para describir 
requerimientos~\cite{vanLamsweerde:2000:HOG:357525.357521}. La motivación para 
utilizar LTL sobre fluents es que 
nos provee de un framework uniforme para especificar propiedades temporales de 
los estados en modelos basados en 
eventos~\cite{vanLamsweerde:2000:HOG:357525.357521}.
FLTL~\cite{FLTL} es una logica lineal temporal para predicar sobre fluents.

\begin{definition}{(Fluent)}
Sea $Acciones$ el conjunto universal de etiquetas de acciones. 
Un fluent es es una tupla $fl = \langle I_{fl}, T_{fl}, Init_{fl} \rangle$ en 
donde $I_{fl} \subseteq Acciones$ son las acciones 
que inicializan, $T_{fl} \subseteq Acciones$ las que terminan, tales que 
$T_{fl} \cap I_{fl} = \emptyset$ y $Init_{fl} \in \{true, false\}$ indica el 
estado inicial del fluent $fl$. Notaremos a $\mathcal{F}$ al universo de todos 
los posibles fluents sobre $Acciones$.
\end{definition}

La logica Fluent Linear Temporal Logic \cite{FLTL} (FLTL) nos permite razonar 
sobre fluents.

\begin{definition}{(Fluent Linear Temporal Logic)}
Sea $Acciones$ el conjunto universal de etiquetas de acciones y $\Pi$ el conjunto de posibles trazas infinitas sobre $Acciones$. 
Una fórmula FLTL se define inductivamente usando los conectores Booleanos standard y los operadores temporales \textbf{X} (Next) y 
\textbf{U} (strong until) de la siguiente forma:

\begin{center}
$\varphi ::= fl \;|\; \neg\varphi \;|\; \varphi \lor \psi \;|\; 
\textbf{X}\varphi \;|\; \varphi\textbf{U}\psi$ 
\end{center}

\end{definition}

donde $fl \in \mathcal{F}$. Dada una traza, por cada posición en la traza 
podemos 
decidir si el fluent está activo o no.

\begin{definition}{(Satisfacción)}
Sea $Acciones$ el conjunto universal de etiquetas de acciones, $Fluents$ el 
conjunto de todos los posibles $\mathcal{F}$ sobre $Acciones$, 
y $\Pi$ el conjunto de posibles trazas infinitas sobre $Acciones$. La traza $\pi \in \Pi$ satisface el fluent $Fl \in Fluents$ en 
la posición $i$ si y sólo si vale alguna de las siguientes condiciones:

\begin{itemize}

\item
$Init_{Fl} \land (\forall j \in \mathbb{N} \cdot 0 \leq j \leq i \rightarrow 
l_{j} \notin T_{Fl})$

\item
$\exists j \in \mathbb{N} \cdot (j \leq i \land l_{j} \in I_{Fl}) \land 
(\forall k \in \mathbb{N} \cdot j \leq k \leq i \rightarrow 	l_{k} \notin 
T_{Fl})$

\end{itemize}

Dada una traza infinita $\pi$, la satisfacción de una fórmula $\varphi$ en la 
posición $i$, denotada $\pi, i\vDash \varphi$ se define segun la 
Figura~\ref{fig:satisfactionop}.

\begin{figure}[b]

$$\begin{array}{lcl}
\pi,i \models_{d} f &\triangleq & \pi,i \models_{d} f  \\
\pi,i \models_{d} \neg \varphi & \triangleq & \neg(\pi,i \models_{d} \varphi) \\
\pi,i \models_{d} \varphi \vee \psi &\triangleq& (\pi,i\models_{d} \varphi) 
\vee (\pi,i \models_{d} 
\psi) \\
\pi,i \models_{d} \mathbf{X} \varphi &\triangleq & \pi,i+1 \models_{d} \varphi 
\\
\pi,i \models_{d} \varphi \mathbf{U} \psi &\triangleq&\exists j \geq i \cdot 
\pi,j \models_{d} \psi 
\quad \wedge \\
& & \qquad \quad \forall \mbox{ }i \leq k < j \cdot \pi,k \models_{d} \varphi\\
\end{array}$$
\caption{Semantics for the satisfaction operator.}
\label{fig:satisfactionop}

\end{figure}

Decimos que $\varphi$ se sostiene en $\pi$ si $\pi, 0\vDash \varphi$. Una fórmula $\varphi \in FLTL$ se sostiene en un LTS $E$ si se sostiene en cada traza
infinita producida por $E$.

\end{definition}


\section{Generalised Reactivity (1)}

Dada una secuencia infinita de estados, una fórmula GR(1) \cite{SRD} denota cuales han ocurrido infinitamente. 
Más formalmente, $\phi = (A, B)$ en donde $A$ y $B$ son conjuntos de subconjuntos de estados en donde o bien alguno de los 
subconjuntos de $A$ no contiene ningún estado que se visita infinitamente, o bien todos los subconjuntos de $B$ contienen 
al menos un estado que se visita infinitamente.

\begin{definition}{(Generalised Reactivity (1))}
Sea $Estados$ el conjunto universal de estados. Dada una secuancia infinita de estados $p$, sean $inf(p)$ los estados que 
ocurren infinitamente en $p$. Sea $\phi_{1}, ..., \phi_{n}$ y $\varUpsilon_{1}, ..., \varUpsilon_{m}$ subconjuntos de $Estados$. Sea\\ 
$gr((\phi_{1}, ..., \phi_{n}), (\varUpsilon_{1}, ..., \varUpsilon_{m}))$ el conjunto de infinitas secuencias $p$ tales que 
o bien para algun $i$ vale $inf(p)\cap\phi_{i} = \varnothing$ o para todo $j$ vale $inf(p)\cap\varUpsilon_{j} \neq \varnothing$.
\end{definition}

\section{Síntesis de controladores}

Dados dos LTSs $N$ y $M$, y $A_{N_{u}} \subseteq A_{N}$, decimos que $M$ es un 
LTS Legal para $N$ respecto a $A_{N_{u}}$ si por cada 
estado en $E\parallel$$M$, realizar una acción perteneciente a $A_{N_{u}}$ es 
igual a realizar la acción en $N$. 
Intuitivamente, en la composición una acción es deshabilitada si y sólo si 
también es deshabilitada en el LTS original. 
En otras palabras, $M$ no restringe a $N$ respecto a $A_{N_{u}}$.

\begin{definition}{(LTS Legal)}
Sean $M = (S_{M}, A_{M}, \Delta_{M}, s_{0}^{M})$ y $N = (S_{N}, A_{N}, 
\Delta_{N}, s_{0}^{N})$ dos LTSs y sea $A_{N_{u}} \in A_{N}$ un estado. 
Decimos que $M$ es un LTS legal para $N$ respecto a $A_{N_{u}}$, si para todo 
($s_{n}, s_{m}) \in E \parallel M$ pasa que 
$\Delta_{E \parallel M}((s_{n}, s_{m})) \cap A_{N_{u}} = \Delta_{N}(s_{N}) \cap 
A_{N_{u}}$.
\end{definition}


La síntesis de controladores se encarga de producir automáticamente una máquina que restringe la ocurrencia de eventos 
controlables basándose en la observación de los eventos que ya ocurrieron. Cuando esta máquina es desplegada en un entorno 
adecuado garantiza la satisfacción de un conjunto de objetivos dado. La satisfacción de estos objetivos depende de la 
satisfacción de las asunciones sobre el entorno.

\subsection{LTS control problem}

Podemos describir la síntesis de controladores de la siguiente forma. Dados un LTS que describe el comportamiento 
del entorno, un conjunto de acciones controlables, un conjunto de fórmulas FLTL que representan las suposiciones 
del dominio y un conjunto de fórmulas FLTL que representan los objetivos del sistema, el problema de control de 
LTS \cite{LTSControl} es encontrar un LTS que sólo restrinja la ocurrencia de acciones controlables y garantice que la composición 
en paralelo entre el entorno y el LTS es deadlock free y que si las suposiciones de dominio se satisfacen, entonces 
los objetivos del sistema también son satisfechos. Intuitivamente, al resolver un problema de síntesis, buscamos evitar alcanzar 
estados para los cuales no se puedan satisfacer los objetivos. Estos estados no deseables los vamos a llamar estados perdedores. 
Las zonas inseguras son conjuntos de estados perdedores.

\begin{definition}{(LTS control)}
Dado un modelo del domino en forma de un LTS determinístico $E = (S_{E}, A_{E}, \Delta_{E}, s_{0}^{E})$, un conjunto de 
acciones controlables $A_{c} \subseteq A_{E}$ y una fórmula FLTL $\varphi$, la solución al LTS control problem 
$\varepsilon = \langle E, \varphi, A_{c} \rangle$ es un LTS $M = (S_{M}, A_{M}, \Delta_{M}, s_{0}^{M})$ tal que 
$A_{M} = A_{E}$, y por cada estado en $S_{M}$ todas las acciones en $A_{M} \setminus A_{c}$ estás habilitadas, $E \parallel M$ 
es deadlock free, y toda traza $\pi$ en $E \parallel M$ garantiza $\pi \vDash \varphi$
\end{definition}

El LTS Control Problem es decidible en tiempo exponencial y en caso de ser decidible, se encuentra la solución con la 
misma complejidad. Para esto es necesario que el modelo sea determinístico.

\subsection{MTS	control problem}
\label{sec:MTS_control}

El problema de la síntesis de controladores para MTSs \cite{MTSControl} consiste en ver si todas, alguna o ninguna de sus 
implementaciones pueden ser controladas por un controlador LTS. Al responder esta pregunta, también estamos respondiendo 
la pregunta de si el problema es realizable, ya que si la respuesta es ninguno el problema no es realizable. Para decidir 
este problema, primero se intenta encontrar un controlador para la implementación del MTS que tiene todas las acciones 
posibles controlables. Si se lo encuentra, se tiene un controlador para todas las posibles implementaciones. Se hace lo 
mismo con el que tiene la mínima cantidad de acciones controlables y así se obtiene la respuesta.

\begin{definition}{(MTS control)}
Dado un modelo del domino en forma de un MTS $M = (S_{M}, A, \Delta_{M}^{r}, \Delta_{M}^{p}, s_{0}^{M})$, un conjunto de 
acciones controlables $A_{c} \subseteq A_{M}$ y una fórmula FLTL $\varphi$, la solución al MTS control problem 
$\varepsilon = \langle M, \varphi, A_{c} \rangle$ es responder:

\begin{itemize}

\item
\textbf{All} si para todo LTS $I$ que sea implementación de $M$, el LTS control problem $\langle I, \varphi, A_{c} \rangle$ es realizable.

\item
\textbf{None} si para ningún LTS $I$ que sea implementación de $M$, el LTS control problem $\langle I, \varphi, A_{c} \rangle$ es realizable.

\item
\textbf{Some} en otro caso.

\end{itemize}

\end{definition}

\section{Finite State Process}
El Finite State Process (FSP) \cite{FSP} es un lenguaje de especificación semántica bien definida en términos de LTSs que nos permite describirlos 
en forma concisa. El lenguaje que utilizaremos es una extensión de FSP que 
puede describir MTSs.